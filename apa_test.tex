\documentclass{stats_apa_style2}

% Copy-paste bibtex format from Google scholar, eazy peazy!
\begin{filecontents*}{MyReferences.bib}
	@article{mcboat2016cool,
		title={Cool title for a paper},
		author={McBoatface, Boaty},
		journal={Journal of Awesome Boatery},
		pages={230},
		year={2016},
		publisher={Oxford University Press}
	}
\end{filecontents*}

\title{Study On Why Fake Statistics Make Good Templates}
\author{Dylan Goldsborough}
\date{\today}
\course{Methods \& Statistics 210}
\instructor{Instructor: Dr. Marcin Sklad}
\university{University College Roosevelt}
\additional{Dylan: Everything\break Not-Dylan: Nothing\break Wordcount: 1,000}
\shorttitle{Study Fake Statistics}

\begin{document}

\maketitle

\begin{abstract}
\noindent
\lipsum[3]
\end{abstract}

\newpage

\section*{Literature Review}

\lipsum[2-3] 
In general, party identification is defined as “a long-term, affective, psychological identification with one’s preferred political party” \cite[p.20]{mcboat2016cool}. This concept is also captured to great extent by the partisanship, the membership of political parties, and the party affiliation, a connection to a party similar to identification. According to \citeA{mcboat2016cool}, this is cool.

\section*{Methods}

\subsection*{Dataset} 

\lipsum[2-3]

\subsection*{Variable Selection}

\lipsum[2-3]

\section*{Results}

\lipsum[2-3]

\section*{Discussion}

\lipsum[2-3]

\newpage

\bibliography{MyReferences}
\bibliographystyle{apacite}

\newpage
\appendix

\section*{Appendix A}
\label{app: A}

\begin{table}[H]
\caption{A table}
\begin{tabular}{@{}*{4}{p{.25\textwidth}@{}}}
\hline 
A & B & C & D \\ 
\hline 
1 & 2 & 3 & 4 \\ 
5 & 8 & 6 & 7 \\ 
9 & 4 & 2 & 1 \\ 
\hline 
\end{tabular} 
\label{table: test}
\end{table}


\newpage

\section*{Appendix B}
\label{app: B}

\begin{figure}[H]
	\caption{A cute kitten}
	\includegraphics[scale=0.5]{kitten.jpg} 
	\label{fig:gray1}
\end{figure}

\newpage

\section*{Appendix C}
\label{app: C}
\setstretch{1.0}
\begin{lstlisting}
import numpy as np
 
def incmatrix(genl1,genl2):
    m = len(genl1)
    n = len(genl2)
    M = None #to become the incidence matrix
    VT = np.zeros((n*m,1), int)  #dummy variable
 
    #compute the bitwise xor matrix
    M1 = bitxormatrix(genl1)
    M2 = np.triu(bitxormatrix(genl2),1) 
 
    for i in range(m-1):
        for j in range(i+1, m):
            [r,c] = np.where(M2 == M1[i,j])
            for k in range(len(r)):
                VT[(i)*n + r[k]] = 1;
                VT[(i)*n + c[k]] = 1;
                VT[(j)*n + r[k]] = 1;
                VT[(j)*n + c[k]] = 1;
 
                if M is None:
                    M = np.copy(VT)
                else:
                    M = np.concatenate((M, VT), 1)
 
                VT = np.zeros((n*m,1), int)
 
    return M
\end{lstlisting}

\end{document} \grid
